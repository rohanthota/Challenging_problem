\documentclass[journal,12pt,twocolumn]{IEEEtran}

\usepackage{setspace}
\usepackage{gensymb}
\singlespacing
\usepackage[cmex10]{amsmath}

\usepackage{amsthm}

\usepackage{mathrsfs}
\usepackage{txfonts}
\usepackage{stfloats}
\usepackage{bm}
\usepackage{cite}
\usepackage{cases}
\usepackage{subfig}

\usepackage{longtable}
\usepackage{multirow}

\usepackage{enumitem}
\usepackage{mathtools}
\usepackage{steinmetz}
\usepackage{tikz}
\usepackage{circuitikz}
\usepackage{verbatim}
\usepackage{tfrupee}
\usepackage[breaklinks=true]{hyperref}
\usepackage{graphicx}
\usepackage{tkz-euclide}

\usetikzlibrary{calc,math}
\usepackage{listings}
    \usepackage{color}                                            %%
    \usepackage{array}                                            %%
    \usepackage{longtable}                                        %%
    \usepackage{calc}                                             %%
    \usepackage{multirow}                                         %%
    \usepackage{hhline}                                           %%
    \usepackage{ifthen}                                           %%
    \usepackage{lscape}     
\usepackage{multicol}
\usepackage{chngcntr}

\DeclareMathOperator*{\Res}{Res}

\renewcommand\thesection{\arabic{section}}
\renewcommand\thesubsection{\thesection.\arabic{subsection}}
\renewcommand\thesubsubsection{\thesubsection.\arabic{subsubsection}}

\renewcommand\thesectiondis{\arabic{section}}
\renewcommand\thesubsectiondis{\thesectiondis.\arabic{subsection}}
\renewcommand\thesubsubsectiondis{\thesubsectiondis.\arabic{sub subsection}}


\hyphenation{optical networks semiconduc-tor}
\def\inputGnumericTable{}                                 %%

\lstset{
%language=C,
frame=single, 
breaklines=true,
columns=fullflexible
}
\date{March 2021}

\begin{document}

\newcommand{\BEQA}{\begin{eqnarray}}
\newcommand{\EEQA}{\end{eqnarray}}
\newcommand{\define}{\stackrel{\triangle}{=}}
\bibliographystyle{IEEEtran}
\raggedbottom
\setlength{\parindent}{0pt}
\providecommand{\mbf}{\mathbf}
\providecommand{\pr}[1]{\ensuremath{\Pr\left(#1\right)}}
\providecommand{\qfunc}[1]{\ensuremath{Q\left(#1\right)}}
\providecommand{\fn}[1]{\ensuremath{f\left(#1\right)}}
\providecommand{\e}[1]{\ensuremath{E\left(#1\right)}}
\providecommand{\sbrak}[1]{\ensuremath{{}\left[#1\right]}}
\providecommand{\lsbrak}[1]{\ensuremath{{}\left[#1\right.}}
\providecommand{\rsbrak}[1]{\ensuremath{{}\left.#1\right]}}
\providecommand{\brak}[1]{\ensuremath{\left(#1\right)}}
\providecommand{\lbrak}[1]{\ensuremath{\left(#1\right.}}
\providecommand{\rbrak}[1]{\ensuremath{\left.#1\right)}}
\providecommand{\cbrak}[1]{\ensuremath{\left\{#1\right\}}}
\providecommand{\lcbrak}[1]{\ensuremath{\left\{#1\right.}}
\providecommand{\rcbrak}[1]{\ensuremath{\left.#1\right\}}}
\theoremstyle{remark}
\newtheorem{rem}{Remark}
\newcommand{\sgn}{\mathop{\mathrm{sgn}}}
\providecommand{\abs}[1]{\vert#1\vert}
\providecommand{\res}[1]{\Res\displaylimits_{#1}} 
\providecommand{\norm}[1]{\lVert#1\rVert}
%\providecommand{\norm}[1]{\lVert#1\rVert}
\providecommand{\mtx}[1]{\mathbf{#1}}
\providecommand{\mean}[1]{E[ #1 ]}
\providecommand{\fourier}{\overset{\mathcal{F}}{ \rightleftharpoons}}
%\providecommand{\hilbert}{\overset{\mathcal{H}}{ \rightleftharpoons}}
\providecommand{\system}{\overset{\mathcal{H}}{ \longleftrightarrow}}
	%\newcommand{\solution}[2]{\textbf{Solution:}{#1}}
\newcommand{\solution}{\noindent \textbf{Solution: }}
\newcommand{\cosec}{\,\text{cosec}\,}
\providecommand{\dec}[2]{\ensuremath{\overset{#1}{\underset{#2}{\gtrless}}}}
\newcommand{\myvec}[1]{\ensuremath{\begin{pmatrix}#1\end{pmatrix}}}
\newcommand{\mydet}[1]{\ensuremath{\begin{vmatrix}#1\end{vmatrix}}}
\numberwithin{equation}{subsection}
\makeatletter
\@addtoreset{figure}{problem}
\makeatother
\let\StandardTheFigure\thefigure
\let\vec\mathbf
\renewcommand{\thefigure}{\theproblem}
\def\putbox#1#2#3{\makebox[0in][l]{\makebox[#1][l]{}\raisebox{\baselineskip}[0in][0in]{\raisebox{#2}[0in][0in]{#3}}}}
     \def\rightbox#1{\makebox[0in][r]{#1}}
     \def\centbox#1{\makebox[0in]{#1}}
     \def\topbox#1{\raisebox{-\baselineskip}[0in][0in]{#1}}
     \def\midbox#1{\raisebox{-0.5\baselineskip}[0in][0in]{#1}}
\vspace{3cm}
\title{AI 1103 - Assignment 5}
\author{T. Rohan \\ CS20BTECH11064}
\maketitle
\newpage
\bigskip
\renewcommand{\thefigure}{\theenumi}
\renewcommand{\thetable}{\theenumi}
Download all latex codes from 
\begin{lstlisting}
https://github.com/rohanthota/Challenging_problem/main.tex
\end{lstlisting}
\section*{\emph{Question}}
Which of the following conditions imply independence of the random variables X
and Y ?
\begin{enumerate}
    \item$\pr{X > a|Y > a} = \pr{X > a} \forall a \in \mathbb{R}$\\ 
    \item$\pr{X > a|Y < b} = \pr{X > a} \forall a, b \in \mathbb{R}$\\ 
    \item$X$ and $Y$ are uncorrelated.\\
    \item$E[(X-a)(Y-b)]=E(X-a) \times E(Y-b) \forall a, b \in \mathbb{R}$\\
\end{enumerate}
\section*{\emph{Solution}}
\begin{enumerate}

\item Let C.D.F's of X, Y and joint C.D.F (X,Y) be represented as
\begin{align}
    F_X(a) &= \pr{X < a}\\
    F_Y(a) &= \pr{Y < a}\\
    F_{X,Y}(a,a) &= \pr{X < a, Y < a}.
\end{align}
To show independence, we want to prove that,
\begin{align}
    F_X(a)F_Y(a) = F_{X, Y}(a,a) \forall a \in \mathbb{R}
\end{align}
Now, we know
\begin{align}
    \pr{X>a|Y>a} &= \frac{\pr{X>a, Y>a}}{\pr{Y>a}}\\
    &= \pr{X>a}\\
    \implies \pr{X>a, Y>a} &= \pr{X>a} \pr{Y>a}
\end{align}
We can write, 
\begin{align}
    \pr{X>a} &= 1 - F_X\label{eq1}\\
    \pr{Y>a} &= 1 - F_Y\label{eq2}
\end{align}
Now, we can write
\begin{align}
    F_Y &= \pr{X<a, Y<a} + \pr{X>a, Y<a}\\
    &= F_{X,Y} + \pr{X>a, Y<a}
\end{align}
\begin{align}
\implies \pr{X>a, Y<a} = F_Y - F_{X,Y}\label{eq3}
\end{align}
Now, 
\begin{math}
\pr{X>a}=\pr{X>a, Y<a}
\end{math}
\begin{align}
    +\pr{X>a, Y>a}
\end{align}
From \eqref{eq1} and \eqref{eq3}
\begin{align}
    1 - F_X = F_Y - F_{X,Y} + \pr{X>a, Y>a}\\
    \implies \pr{X>a, Y>a} = 1 - F_X - F_Y + F_{X,Y}.\label{eq4}
\end{align}
From \eqref{eq1}, \eqref{eq2} and \eqref{eq4}
\begin{align}
    1 - F_X - F_Y + F_{X,Y} &= (1 - F_X) \times (1 - F_Y)\\
    \implies F_{X, Y}(a,a) &= F_X(a)F_Y(a).
\end{align}
Therefore, option 1 is correct.
    
\item Similar to option 1, C.D.F.s are represented as below,\\
\begin{align}
    F_X(a) &= \pr{X < a}\\
    F_Y(b) &= \pr{Y < b}\\
    F_{X,Y}(a,b) &= \pr{X < a, Y < b}\label{eq7}
\end{align}
To show independence, we want to prove that,
\begin{align}
    F_X(a)F_Y(b) = F_{X, Y}(a,b) \forall a, b \in \mathbb{R}\label{eq5}
\end{align}
From conditional probability we know: 
\begin{align}
    \pr{X > a | Y < b} = \frac{\pr{X > a, Y < b}}{\pr{Y < b}}
\end{align}
and so using the given condition in option 2), we can write \eqref{eq5} as
\begin{align}
    \pr{X > a} &= \frac{\pr{X > a, Y < b}}{\pr{Y < b}}\\
    \pr{X > a}\pr{Y < b} &= \pr{X > a, Y < b}\label{eq6}
\end{align}
We can write the C.D.F as 
\begin{align}
    \pr{X > a} &= 1 - F_X(a)\label{eq9}\\
    \pr{Y < b} &= F_Y(b)\label{eq10}\\
\end{align}
\begin{math}
    F_Y(b) = \pr{X > a, Y < b} +
\end{math}
\begin{align}
    \pr{X < a, Y < b}
\end{align}
From \eqref{eq7}
\begin{align}
    \pr{X > a, Y < b} = F_Y(b) - F_{X, Y}(a,b)\label{eq8}
\end{align}
From \eqref{eq9}, \eqref{eq10} and \eqref{eq8}
\begin{align}
    (1 - F_X(a))(F_Y(b)) &= F_Y(b) - F_{X, Y}(a,b)\\
    \implies F_X(a)F_Y(b) &= F_{X, Y}(a,b)
\end{align}
Therefore, option 2 is correct.

\item Given random variables X and Y are uncorrelated which means that their correlation is 0, or, equivalently, Cov(X, Y) = 0.
\begin{align}
    Cov(X, Y) &= E\brak{XY} - E\brak{X}E\brak{Y}\\
    &[\because Cov(X, Y) = 0]\\
    E\brak{XY} &= E\brak{X}E\brak{Y}
\end{align}
Trying to solve through a counter example,
Let there be three random variables X, Y and Z such that,
\begin{align}
    &Z \sim \mathcal{N}(0, 1)\\
    &X = Z\\
    &Y = Z^2
\end{align}
They are clearly independent, now let's check if they are uncorrelated
\begin{align}
  E\brak{XY} = E\brak{Z^3} = 0\\
  E\brak{X}E\brak{Y} = E\brak{Z}E\brak{Z^2} = 0\\
  \implies E\brak{XY} = E\brak{X}E\brak{Y}
\end{align}
Therefore, they are dependent and uncorrelated. Hence, option 3 is incorrect.
\item We extend L.H.S
\begin{math}
    E[(X-a)(Y-b)] = E[XY]
\end{math}
\begin{align}
    - aE[Y] - bE[X] + ab\label{eq12}
\end{align}
and R.H.S to compare,
\begin{math}
    E(X-a) E(Y-b) = E[X]E[Y]
\end{math}
\begin{align}
    - aE[Y] - bE[X] + ab\label{eq13}
\end{align}
From \eqref{eq12} and \eqref{eq13}
\begin{align}
    E[XY] = E[X] E[Y]
\end{align}
 Which was already dealt in option 3, Hence option 4 is incorrect.
\end{enumerate}
\end{document}
