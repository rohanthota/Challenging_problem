\documentclass[journal,12pt,twocolumn]{IEEEtran}

\usepackage{setspace}
\usepackage{gensymb}
\singlespacing
\usepackage[cmex10]{amsmath}

\usepackage{amsthm}

\usepackage{mathrsfs}
\usepackage{txfonts}
\usepackage{stfloats}
\usepackage{bm}
\usepackage{cite}
\usepackage{cases}
\usepackage{subfig}

\usepackage{longtable}
\usepackage{multirow}
\usepackage{physics}
\usepackage{enumitem}
\usepackage{mathtools}
\usepackage{steinmetz}
\usepackage{tikz}
\usepackage{circuitikz}
\usepackage{verbatim}
\usepackage{tfrupee}
\usepackage[breaklinks=true]{hyperref}
\usepackage{graphicx}
\usepackage{tkz-euclide}

\usetikzlibrary{calc,math}
\usepackage{listings}
    \usepackage{color}                                            %%
    \usepackage{array}                                            %%
    \usepackage{longtable}                                        %%
    \usepackage{calc}                                             %%
    \usepackage{multirow}                                         %%
    \usepackage{hhline}                                           %%
    \usepackage{ifthen}                                           %%
    \usepackage{lscape}     
\usepackage{multicol}
\usepackage{chngcntr}



\renewcommand\thesection{\arabic{section}}
\renewcommand\thesubsection{\thesection.\arabic{subsection}}
\renewcommand\thesubsubsection{\thesubsection.\arabic{subsubsection}}

\renewcommand\thesectiondis{\arabic{section}}
\renewcommand\thesubsectiondis{\thesectiondis.\arabic{subsection}}
\renewcommand\thesubsubsectiondis{\thesubsectiondis.\arabic{sub subsection}}


\hyphenation{optical networks semiconduc-tor}
\def\inputGnumericTable{}                                 %%

\lstset{
%language=C,
frame=single, 
breaklines=true,
columns=fullflexible
}
\date{March 2021}

\begin{document}

\newcommand{\BEQA}{\begin{eqnarray}}
\newcommand{\EEQA}{\end{eqnarray}}
\newcommand{\define}{\stackrel{\triangle}{=}}
\bibliographystyle{IEEEtran}
\raggedbottom
\setlength{\parindent}{0pt}
\providecommand{\mbf}{\mathbf}
\providecommand{\pr}[1]{\ensuremath{\Pr\left(#1\right)}}
\providecommand{\qfunc}[1]{\ensuremath{Q\left(#1\right)}}
\providecommand{\fn}[1]{\ensuremath{f\left(#1\right)}}
\providecommand{\e}[1]{\ensuremath{E\left(#1\right)}}
\providecommand{\sbrak}[1]{\ensuremath{{}\left[#1\right]}}
\providecommand{\lsbrak}[1]{\ensuremath{{}\left[#1\right.}}
\providecommand{\rsbrak}[1]{\ensuremath{{}\left.#1\right]}}
\providecommand{\brak}[1]{\ensuremath{\left(#1\right)}}
\providecommand{\lbrak}[1]{\ensuremath{\left(#1\right.}}
\providecommand{\rbrak}[1]{\ensuremath{\left.#1\right)}}
\providecommand{\cbrak}[1]{\ensuremath{\left\{#1\right\}}}
\providecommand{\lcbrak}[1]{\ensuremath{\left\{#1\right.}}
\providecommand{\rcbrak}[1]{\ensuremath{\left.#1\right\}}}
\theoremstyle{remark}
\newtheorem{rem}{Remark}
\newcommand{\sgn}{\mathop{\mathrm{sgn}}}
\providecommand{\abs}[1]{\vert#1\vert}
\providecommand{\res}[1]{\Res\displaylimits_{#1}} 
\providecommand{\norm}[1]{\lVert#1\rVert}
%\providecommand{\norm}[1]{\lVert#1\rVert}
\providecommand{\mtx}[1]{\mathbf{#1}}
\providecommand{\mean}[1]{E[ #1 ]}
\providecommand{\fourier}{\overset{\mathcal{F}}{ \rightleftharpoons}}
%\providecommand{\hilbert}{\overset{\mathcal{H}}{ \rightleftharpoons}}
\providecommand{\system}{\overset{\mathcal{H}}{ \longleftrightarrow}}
	%\newcommand{\solution}[2]{\textbf{Solution:}{#1}}
\newcommand{\solution}{\noindent \textbf{Solution: }}
\newcommand{\cosec}{\,\text{cosec}\,}
\providecommand{\dec}[2]{\ensuremath{\overset{#1}{\underset{#2}{\gtrless}}}}
\newcommand{\myvec}[1]{\ensuremath{\begin{pmatrix}#1\end{pmatrix}}}
\newcommand{\mydet}[1]{\ensuremath{\begin{vmatrix}#1\end{vmatrix}}}
\numberwithin{equation}{subsection}
\makeatletter
\@addtoreset{figure}{problem}
\makeatother
\let\StandardTheFigure\thefigure
\let\vec\mathbf
\renewcommand{\thefigure}{\theproblem}
\def\putbox#1#2#3{\makebox[0in][l]{\makebox[#1][l]{}\raisebox{\baselineskip}[0in][0in]{\raisebox{#2}[0in][0in]{#3}}}}
     \def\rightbox#1{\makebox[0in][r]{#1}}
     \def\centbox#1{\makebox[0in]{#1}}
     \def\topbox#1{\raisebox{-\baselineskip}[0in][0in]{#1}}
     \def\midbox#1{\raisebox{-0.5\baselineskip}[0in][0in]{#1}}
\vspace{3cm}
\title{AI 1103 - Challenging Problem 11}
\author{T. Rohan \\ CS20BTECH11064}
\maketitle
\newpage
\bigskip
\renewcommand{\thefigure}{\theenumi}
\renewcommand{\thetable}{\theenumi}
Download all latex codes from 
\begin{lstlisting}
https://github.com/rohanthota/Challenging_problem/main.tex
\end{lstlisting}
\section{Problem}
(UGC/MATH 2018 (June set-a)-Q.106) Let ${X_i}_{i \geq 1}$ be a sequence of i.i.d. random variables with $E(X_i)=0$ and $V(X_i)=1$. Which of the following are true?
\vspace{0.2cm}
\begin{enumerate}
    \item $\dfrac{1}{n} \sum_{i=1}^n X_i^2 \to 0$ in probability 
    \item $\dfrac{1}{n^{3/4}} \sum_{i=1}^n X_i \to 0$ in probability 
    \item $\dfrac{1}{n^{1/2}} \sum_{i=1}^n X_i \to 0$ in probability 
    \item $\dfrac{1}{n} \sum_{i=1}^n X_i^2 \to 1$ in probability
\end{enumerate}

\section*{\emph{Solution}}
CONVERGENCE IN DISTRIBUTION :

A sequence of random variables $Y$, $Y_1$, $Y_2 \ldots$   converges in distribution to a random variable $Y$, shown by {$Y_n$} \to Y , \text{if}

\begin{align}
    \lim_{n \to \infty}F_{X_{n}} (a) = F_{X} (a)  \text{  }\forall a \in \mathbb{R}.
\end{align}

CONVERGENCE IN PROBABILITY:

A sequence of random variables $Y$, $Y_1$, $Y_2 \ldots$ is said to converge in probability to $Y$, if

\begin{align}
    \lim_{n \to \infty}\pr{\abs{Y_n - Y} > \epsilon} = 0  \text{  }\forall \epsilon > 0.
\end{align}




THEOREM : 
If
\begin{math}
 {Y_n} \to Y \text{ in probability, }{Y_n} \to Y \text{ in distribution.}
\end{math}



CENTRAL LIMIT THEOREM(CLT):

Let $X_1$, $X_2$, \ldots $X_n$ be i.i.d. random variables with expected value $E(X_i)=\mu < \infty$  and $0 < V(X_i)=\sigma^2 < \infty$. Then the random variable 

\begin{align}
    Z_n = \frac{\bar{x} - \mu}{\frac{\sigma}{\sqrt{n}}} = \frac{X_1 + X_2 + \ldots + X_n - n\mu}{\sqrt{n}\sigma}
\end{align}

converges in distribution to the standard normal random variable as n goes to infinity, that is
\begin{align}
    \lim_{n \to \infty}\pr{Z_n \leq a} = \Phi(a)   \text{        }\forall a \in \mathbb{R}.
\end{align}
where $\Phi(a)$ is the standard normal CDF.

\begin{enumerate}
\setcounter{enumi}{2}
    
\item The option states that
\begin{math}
 \dfrac{1}{n^{1/2}} \sum_{i=1}^n X_i \to 0
\end{math} 
in probability. This statement implies that $\dfrac{1}{n^{1/2}} \sum_{i=1}^n X_i \to 0$ in distribution, from the theorem mentioned above.

Writing the random variable $Z_n$ from CLT, for the i.i.d random variables,

    \begin{align}
        Z_n &= \frac{X_1 + X_2 + \ldots + X_n - n\mu}{\sqrt{n}\sigma}\\
        &= \frac{X_1 + X_2 + \ldots + X_n}{\sqrt{n}}\\
        &= \dfrac{1}{n^{1/2}} \sum_{i=1}^n X_i
    \end{align}
Since $\mu = 0$ and $\sigma = 1$.

According to Central limit theorem, 
\begin{align}
    Z_n \to Z, \text{where, } Z \sim N(0,1)
\end{align}
which is not what the option states. Therefore, option 3 is incorrect.
\end{enumerate}
\end{document}
